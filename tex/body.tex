
\documentclass[12pt,a4paper,openright,twoside]{report}

\usepackage[italian]{babel}
\usepackage[utf8x]{inputenc} % per i caratteri è à ò , only Linux
\usepackage{fancyhdr}
\usepackage{indentfirst} % indentazione all'inizio dei capitoli
%\usepackage{showkeys}
\usepackage{graphicx}
\usepackage{newlfont}
% librerie matematiche
\usepackage{amssymb}
\usepackage{amsmath}
\usepackage{latexsym}
\usepackage{amsthm}
% altre
\usepackage{import}
\usepackage{listings}
\usepackage{graphicx}
\usepackage{verbatim}
\usepackage{color}
\usepackage{titlesec} % per paragrafi
\usepackage[normalem]{ulem}
\usepackage{tabularx}
\usepackage{newlfont}
\useunder{\uline}{\ul}{}

\graphicspath{ {img/} }

%
\oddsidemargin=30pt \evensidemargin=20pt
\hyphenation{sil-la-ba-zio-ne pa-ren-te-si}
%serve per la sillabazione: tra parentesi vanno inserite come nell'esempio le parole 
%che latex non riesce a tagliare nel modo giusto andando a capo.

% comandi per l'impostazione della pagina, vedi man fancyhdr
\pagestyle{fancy}\addtolength{\headwidth}{20pt}
\renewcommand{\chaptermark}[1]{\markboth{\thechapter.\ #1}{}}
\renewcommand{\sectionmark}[1]{\markright{\thesection \ #1}{}}
\rhead[\fancyplain{}{\bfseries\leftmark}]{\fancyplain{}{\bfseries\thepage}}
\cfoot{}

\linespread{1.3} 

%
\begin{document}

\begin{titlepage}
\begin{center}
\textwidth=450pt\oddsidemargin=0pt
{{\Large{\textsc{Alma Mater Studiorum $\cdot$ Universit\`a di
Bologna}}}} \rule[0.1cm]{15.8cm}{0.1mm}
\rule[0.5cm]{15.8cm}{0.6mm}
{\small{\bf SCUOLA DI SCIENZE\\
Corso di Laurea in Informatica }}
\end{center}
\vspace{15mm}

\begin{center}
{\LARGE{\bf Studio e sperimentazione di strumenti software per presentazioni pubbliche basati su Raspberry Pi}}\\
\vspace{3mm}
\end{center}

\vspace{40mm}
\par
\noindent
\begin{minipage}[t]{0.47\textwidth}
{\large{\bf Relatore:\\
Chiar.mo Prof.\\
Renzo Davoli}}
\end{minipage}
\hfill
\begin{minipage}[t]{0.47\textwidth}\raggedleft
{\large{\bf Presentata da:\\
Giulia Cantini}}
\end{minipage}
\vspace{20mm}
\begin{center}
{\large{\bf Sessione II\\%inserire il numero della sessione in cui ci si laurea
Anno Accademico 2016/2017}}%inserire l'anno accademico a cui si è iscritti
\end{center}
\end{titlepage}

%\pagenumbering{roman} %numerazione pagina con numeri romani
%\pagenumbering{arabic}
\addcontentsline{toc}{chapter}{Introduzione}
\chapter*{Introduzione} % per avere un capitolo non numerato inserire un * tra chapter e le parentesi
\pagenumbering{arabic}
\import{tex/}{introduzione}

\clearpage{\pagestyle{empty}\cleardoublepage}
\tableofcontents % crea l'indice
% imposta l'intestazione di pagina
\rhead[\fancyplain{}{\bfseries\leftmark}]{\fancyplain{}{\bfseries\thepage}}
\lhead[\fancyplain{}{\bfseries\thepage}]{\fancyplain{}{\bfseries
INDICE}}
\clearpage{\pagestyle{empty}\cleardoublepage} % non numera l'ultima pagina sinistra
\listoffigures  % crea l'elenco delle figure
\clearpage{\pagestyle{empty}\cleardoublepage}
\listoftables  % crea l'elenco delle tabelle
\clearpage{\pagestyle{empty}\cleardoublepage} % vedi sopra

%%%%%%%%%%CONTESTO%%%%%%%%%%%%%%
%\pagenumbering{arabic}
\chapter{Contesto} % crea nuovo capitolo Contesto
%imposta l'intestazione di pagina
\lhead[\fancyplain{}{\bfseries\thepage}]{\fancyplain{}{\bfseries\rightmark}}
%\pagenumbering{arabic} % mette i numeri arabi

\import{tex/}{contesto} 

\clearpage{\pagestyle{empty}\cleardoublepage} % non numera l'ultima pagina sinistra

%%%%%%%%%%%%% LAVORO ORIGINALE PRIMA PARTE %%%%%%%%%%%%%%%%%%%%%%%%%%%%%%%%%%%%
\chapter{Lavoro originale: Wireless Projector}

\import{tex/}{lavoro1}

%%%%%%%% SECONDA PARTE %%%%%%%%%
\chapter{Lavoro originale: VNC on the GO}

\import{tex/}{lavoro2}
%%%%%%%%%%%%% VALUTAZIONI E FUTURI SVILUPPI %%%%%%%%%%%%%%%%%%%%%%%%%%%%%%%%%%%
\chapter{Valutazioni e futuri sviluppi}

\import{tex/}{valutazioni}
%%%%%%%%%%%%% CONCLUSIONI %%%%%%%%%%%%%%%%%%%%%%%%%%%%%%%%%%%%%%%%%%%%%%

% per fare le conclusioni, una sorta di recap di quel che si è fatto, in teoria dovrebbe riprendere quanto detto in introduzione
\addcontentsline{toc}{chapter}{Conclusioni}
\chapter*{Conclusioni}
\import{tex/}{conclusioni}


%%%%%%%%%%%%% APPENDICI %%%%%%%%%%%%%%%%%%%%%%%%%%%%%%%%%%%
\begin{comment}
\renewcommand{\chaptermark}[1]{\markright{\thechapter \ #1}{}}
\lhead[\fancyplain{}{\bfseries\thepage}]{\fancyplain{}{\bfseries\rightmark}}
\appendix % imposta le appendici
\chapter{Appendice} % crea l'appendice, come si toglie la A?

\rhead[\fancyplain{}{\bfseries \thechapter \:Appendice}]
{\fancyplain{}{\bfseries\thepage}}

\end{comment}
%%%%%%%%%%%%%% BIBLIOGRAFIA %%%%%%%%%%%%%%%%%%%%%%%%%%%%%%%%%%
\begin{thebibliography}{90} % crea l'ambiente bibliografia
\rhead[\fancyplain{}{\bfseries \leftmark}]{\fancyplain{}{\bfseries
\thepage}}

\addcontentsline{toc}{chapter}{Bibliografia} % aggiunge la bibliografia all'indice
%%%%%%%%%%%%%%%%%%%%%%%%%%%%%%%%%%%%%%%%%provare anche questo comando:
%%%%%%%%%%%\addcontentsline{toc}{chapter}{\numberline{}{Bibliografia}}

\bibitem{K} Larry L. Peterson, Bruce S. Davie. "Computer Networks: A Systems Approach" . Morgan Kaufmann, 2011.
\bibitem{K} Abraham Silberschatz, Peter B. Galvin, Greg Gagne. "Operating System Concepts". John Wiley \& Sons, Inc., 2013. 
\bibitem{K} "Wireless Projector Guide". https://airtame.com/wireless-projector. Ultima visita: 15 settembre 2017.
\bibitem{K} "x11vnc: a VNC server for real X displays". http://www.karlrunge.com/x11vnc/x11vnc\_opts.html. Ultima visita: 15 settembre 2017 .
\bibitem{K} "Raspberry Pi Documentation". https://www.raspberrypi.org/
documentation/. Ultima visita: 15 settembre 2017.
\bibitem{K} "Debian Wiki". https://wiki.debian.org/it/FrontPage. Ultima visita: 15 settembre 2017.
\bibitem{K} "Desktop entries". https://wiki.archlinux.org/index.php/desktop\_entries. Ultima visita: 15 settembre 2017.
\bibitem{K} "PulseAudio/Examples". https://wiki.archlinux.org/index.php/
PulseAudio/Examples\_(Italiano). Ultima visita: 15 settembre 2017.
\bibitem{K} "RaspberryPI in Readonly". https://guglio.xyz/raspberrypi-in-readonly/. Ultima visita: 15 settembre 2017.
\bibitem{K} "Make Raspbian System Read-Only". http://blog.pi3g.com/2014/04/make-raspbian-system-read-only/. Ultima visita: 15 settembre 2017.
\bibitem{K} "Tornado". http://www.tornadoweb.org/en/stable/ . Ultima visita: 15 settembre 2017.
\bibitem{K} "Improve boot performance". https://wiki.archlinux.org/index.php/
Improve\_boot\_performance\_(Italiano). Ultima visita: 15 settembre 2017.

%% quando hai finito con l'ordine metti k1, k2, etc..
\end{thebibliography}
\clearpage{\pagestyle{empty}\cleardoublepage}
\chapter*{Ringraziamenti}
\thispagestyle{empty}
Ringrazio la mia famiglia che mi ha sostenuto moralmente durante questo faticoso percorso durato tre anni.\\
Ringrazio Lorenzo che mi sopporta ogni giorno e condivide con me le gioie e i dolori della vita universitaria.\\
Ringrazio gli amici e i compagni di corso che mi hanno generosamente aiutato nelle diverse occasioni.\\
Infine ringrazio il professore Renzo Davoli per il tempo dedicatomi e i gentili consigli.\\

\end{document}
