Negli ultimi anni si è verificata un rapida diffusione di proiettori wireless, che sostituiscono quelli cablati, offrendo maggiore semplicità d'uso in molteplici contesti, in ambito lavorativo, per presentazioni aziendali o pubbliche, durante conferenze, eventi, o lezioni, oppure in ambiente home con lo streaming di contenuti multimediali per intrattenimento.\\
Uno dei difetti principali di alcuni dei modelli proposti è quello di fondarsi sull'installazione di driver proprietari e \textit{platform-dependent} o sull'aggiunta di dispositivi hardware quali ad esempio dongle WiFi, dai quali dipende strettamente il funzionamento del proiettore. \\
Nel presente progetto viene realizzato un sistema di supporto alla proiezione basato sulla tecnologia VNC (Virtual Network Computing), open source e disponibile in svariate implementazioni su diverse piattaforme al fine di consentire, per quanto possibile, la stessa esperienza di utilizzo del sistema su dispositivi eterogenei.\\
Essenzialmente ciò che questo lavoro si propone di fare, è convertire un proiettore tradizionale in uno dotato di capacità wireless, in modo da ottenere un apparato che sia, al netto dei compromessi, comparabile in termini di efficienza e efficacia a uno dei modelli disponibili in commercio, con particolare attenzione data all'utilizzo del sistema durante presentazioni pubbliche o lezioni.\\
Per fare ciò, l'altra limitazione imposta, oltre alla compatibilità massima, è stata mantenere i costi di realizzazione più bassi possibile: per questa ragione l'hardware utilizzato è costituito da Raspberry Pi, in grado di aumentare le potenzialità del proiettore con costi contenuti.\\
Nella seconda parte del progetto si è posta poi l'enfasi nel realizzare un meccanismo di proiezione ancora più facile ed immediato, basato sulla configurazione ad hoc di un dispositivo (Raspberry Pi Zero W) e mirato alla esclusiva proiezione di documenti PDF.